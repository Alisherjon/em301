\documentclass[pdftex,12pt,a4paper]{article}

% jan 2012


\usepackage[paper=a4paper,top=13.5mm, bottom=13.5mm,left=16.5mm,right=13.5mm,includefoot]{geometry}

\usepackage{etex} % расширение классического tex
% в частности позволяет подгружать гораздо больше пакетов, чем мы и займёмся далее


% sudo yum install texlive-bbm texlive-bbm-macros texlive-asymptote texlive-cm-super texlive-cyrillic texlive-pgfplots texlive-subfigure
% yum install texlive-chessboard texlive-skaknew % for \usepackage{chessboard}
% yum install texlive-minted texlive-navigator texlive-yax texlive-texapi

% растягиваем границы страницы
%\emergencystretch=2em \voffset=-2cm \hoffset=-1cm
%\unitlength=0.6mm \textwidth=17cm \textheight=25cm

\usepackage{makeidx} % для создания предметных указателей
\usepackage{verbatim} % для многострочных комментариев
%\usepackage{cmap} % для поиска русских слов в pdf --- устарело
%\usepackage[pdftex]{graphicx} % для вставки графики 
% omit pdftex option if not using pdflatex


%\usepackage{dsfont} % шрифт для единички с двойной палочкой (для индикатора события)
\usepackage{bbm} % шрифт - двойные буквы

\usepackage[colorlinks,hyperindex,unicode,breaklinks]{hyperref} % гиперссылки в pdf


\usepackage[utf8]{inputenc} % выбор кодировки файла
\usepackage[T2A]{fontenc} % кодировка шрифта
\usepackage[russian]{babel} % выбор языка

\usepackage{amssymb}
\usepackage{amsmath}
\usepackage{amsthm}
\usepackage{epsfig}
\usepackage{bm}
\usepackage{color}

\usepackage[usenames, dvipsnames, svgnames, table, rgb]{xcolor}
\usepackage{colortbl}

\usepackage{multicol}
\usepackage{multirow} % Слияние строк в таблице
\usepackage{dcolumn} % apsr table requirement

\usepackage{textcomp}  % Чтобы в формулах можно было русские буквы писать через \text{}

\usepackage{embedfile} % Чтобы код LaTeXа включился как приложение в PDF-файл

\usepackage{subfigure} % для создания нескольких рисунков внутри одного

\usepackage{tikz,pgfplots} % язык для рисования графики из latex'a
\usetikzlibrary{trees} % прибамбас в нем для рисовки деревьев
\usetikzlibrary{arrows} % прибамбас в нем для рисовки стрелочек подлиннее
\usepackage{tikz-qtree} % прибамбас в нем для рисовки деревьев
\pgfplotsset{compat=1.8}

%\usepackage{verse} % для стихов




\usepackage{enumitem}


\embedfile[desc={Исходный LaTeX файл}]{\jobname.tex} % Включение кода в выходной файл
\embedfile[desc={Стилевой файл}]{title_bor_utf8.tex}



% вместо горизонтальной делаем косую черточку в нестрогих неравенствах
\renewcommand{\le}{\leqslant}
\renewcommand{\ge}{\geqslant} 
\renewcommand{\leq}{\leqslant}
\renewcommand{\geq}{\geqslant}

% делаем короче интервал в списках 
\setlength{\itemsep}{0pt} 
\setlength{\parskip}{0pt} 
\setlength{\parsep}{0pt}

% свешиваем пунктуацию (т.е. знаки пунктуации могут вылезать за правую границу текста, при этом текст выглядит ровнее)
\usepackage{microtype}

% более красивые таблицы
\usepackage{booktabs}
% заповеди из докупентации: 
% 1. Не используйте вертикальные линни
% 2. Не используйте двойные линии
% 3. Единицы измерения - в шапку таблицы
% 4. Не сокращайте .1 вместо 0.1
% 5. Повторяющееся значение повторяйте, а не говорите "то же"


% DEFS
\def \mbf{\mathbf}
\def \msf{\mathsf}
\def \mbb{\mathbb}
\def \tbf{\textbf}
\def \tsf{\textsf}
\def \ttt{\texttt}
\def \tbb{\textbb}

\def \wh{\widehat}
\def \wt{\widetilde}
\def \ni{\noindent}
\def \ol{\overline}
\def \cd{\cdot}
\def \bl{\bigl}
\def \br{\bigr}
\def \Bl{\Bigl}
\def \Br{\Bigr}
\def \fr{\frac}
\def \bs{\backslash}
\def \lims{\limits}
\def \arg{{\operatorname{arg}}}
\def \dist{{\operatorname{dist}}}
\def \VC{{\operatorname{VCdim}}}
\def \card{{\operatorname{card}}}
\def \sgn{{\operatorname{sign}\,}}
\def \sign{{\operatorname{sign}\,}}
\def \xfs{(x_1,\ldots,x_{n-1})}
\def \Tr{{\operatorname{\mbf{Tr}}}}
\DeclareMathOperator*{\argmin}{arg\,min}
\DeclareMathOperator*{\argmax}{arg\,max}
\DeclareMathOperator*{\amn}{arg\,min}
\DeclareMathOperator*{\amx}{arg\,max}
\def \cov{{\operatorname{Cov}}}
\DeclareMathOperator{\Var}{Var}
\DeclareMathOperator{\Cov}{Cov}
\DeclareMathOperator{\Corr}{Corr}
\DeclareMathOperator{\trace}{tr}
%\DeclareMathOperator{\tr}{tr}
\DeclareMathOperator{\rank}{rank}
\DeclareMathOperator{\rk}{rank}


\def \xfs{(x_1,\ldots,x_{n-1})}
\def \ti{\tilde}
\def \wti{\widetilde}


\def \mL{\mathcal{L}}
\def \mW{\mathcal{W}}
\def \mH{\mathcal{H}}
\def \mC{\mathcal{C}}
\def \mE{\mathcal{E}}
\def \mN{\mathcal{N}}
\def \mA{\mathcal{A}}
\def \mB{\mathcal{B}}
\def \mU{\mathcal{U}}
\def \mV{\mathcal{V}}
\def \mF{\mathcal{F}}

\def \R{\mbb R}
\def \N{\mbb N}
\def \Z{\mbb Z}
\def \P{\mbb{P}}
%\def \p{\mbb{P}}
\def \E{\mbb{E}}
\def \D{\msf{D}}
\def \I{\mbf{I}}

\def \a{\alpha}
\def \b{\beta}
\def \t{\tau}
\def \dt{\delta}
\def \e{\varepsilon}
\def \ga{\gamma}
\def \kp{\varkappa}
\def \la{\lambda}
\def \sg{\sigma}
\def \sgm{\sigma}
\def \tt{\theta}
\def \ve{\varepsilon}
\def \Dt{\Delta}
\def \La{\Lambda}
\def \Sgm{\Sigma}
\def \Sg{\Sigma}
\def \Tt{\Theta}
\def \Om{\Omega}
\def \om{\omega}


\def \ni{\noindent}
\def \lq{\glqq}
\def \rq{\grqq}
\def \lbr{\linebreak}
\def \vsi{\vspace{0.1cm}}
\def \vsii{\vspace{0.2cm}}
\def \vsiii{\vspace{0.3cm}}
\def \vsiv{\vspace{0.4cm}}
\def \vsv{\vspace{0.5cm}}
\def \vsvi{\vspace{0.6cm}}
\def \vsvii{\vspace{0.7cm}}
\def \vsviii{\vspace{0.8cm}}
\def \vsix{\vspace{0.9cm}}
\def \VSI{\vspace{1cm}}
\def \VSII{\vspace{2cm}}
\def \VSIII{\vspace{3cm}}


\newcommand{\grad}{\mathrm{grad}}
\newcommand{\dx}[1]{\,\mathrm{d}#1} % для интеграла: маленький отступ и прямая d
\newcommand{\ind}[1]{\mathbbm{1}_{\{#1\}}} % Индикатор события
%\renewcommand{\to}{\rightarrow}
\newcommand{\eqdef}{\mathrel{\stackrel{\rm def}=}}
\newcommand{\iid}{\mathrel{\stackrel{\rm i.\,i.\,d.}\sim}}
\newcommand{\const}{\mathrm{const}}

%на всякий случай пока есть
%теоремы без нумерации и имени
%\newtheorem*{theor}{Теорема}

%"Определения","Замечания"
%и "Гипотезы" не нумеруются
\newtheorem*{definition}{Определение}
%\newtheorem*{rem}{Замечание}
%\newtheorem*{conj}{Гипотеза}

%"Теоремы" и "Леммы" нумеруются
%по главам и согласованно м/у собой
\newtheorem{theorem}{Теорема}
%\newtheorem{lemma}[theorem]{Лемма}

% Утверждения нумеруются по главам
% независимо от Лемм и Теорем
%\newtheorem{prop}{Утверждение}
%\newtheorem{cor}{Следствие}


\title{С 1 апреля!}
\author{}
\date{}

\begin{document}
\maketitle
\parindent=0 pt % отступ равен 0



\begin{enumerate}

\item Рождается старичком, умирает младенцем, сегодня празднует день рождения, но не Гоголь. 
Кто это? Опишите внешний вид, характер, или нарисуйте его :)

\item Для борьбы с гетероскедастичностью в модели $y_i=\beta_1+\beta_2 x_i+\varepsilon_i$ исследователь перешёл к модели $\tilde{y}_i=\beta_1 \frac{1}{z_i}+\beta_2 \tilde{x}_i+\tilde{\varepsilon}_i$, где $\tilde{x}_i=x_i/z_i$, $\tilde{y}_i=y_i/z_i$, $\tilde{\varepsilon}_i=\varepsilon_i/z_i$. 

Какой вид гетероскедастичности предполагался?

\item Василий Аспушкин провёл два разных теста на гетероскедастичность на одном уровне значимости. Оказалось, что в одном из них $H_0$ отвергается, а в другом --- нет. 
\begin{enumerate}
\item Почему это могло случиться?
\item Какой же вывод о гетероскедастичности следует сделать Василию? Что можно сказать об уровне значимости предложенного Вами способа сделать вывод?
 
\end{enumerate}


\item Писатель Василий Аспушкин пишет Большой Роман. Количество страниц, которое он пишет ежедневно, зависит от количества съеденных пирожков, выпитого лимонада и числа посещений Музы. 
\[
Stranitsi_i = \beta_1 + \beta_2 Pirojki_i + \beta_3 Limonad_i + \beta_4 Musa_i + \varepsilon_i
\]

Когда идёт дождь, Василий Аспушкин очень волнуется: он ошибочно считает, что музы плохо летают в дождь. Поэтому в дождливые дни дисперсия $\varepsilon_i$ может быть выше. 


\begin{enumerate}
\item Отсортировав имеющиеся наблюдения по количеству осадков в день, Настойчивый издатель построил регрессию по 40 самым дождливым дням и получил $RSS=\sum_i (y_i-\hat{y}_i)^2=360$. В регрессии по 40 самым сухим дням $RSS=252$. Всего имеется 100 наблюдений. Проверьте гипотезу о гомоскедастичности. Как называется соответствующий тест?

\item Василий Аспушкин оценил по 100 наблюдениям исходную модель с помощью МНК. А затем построил регрессию квадратов стьюдентизированных остатков на количество осадков и константу. Во второй регрессии $R^2=0.3$. Проверьте гипотезу о гомоскедастичности. 

\item Предположим, что дисперсия ошибок линейно зависит от количества осадков. 
\begin{enumerate}
\item Как будет выглядеть функция максимального правдоподобия для оценивания коэффициентов исходной модели?
\item Опишите процедуру доступного обобщенного метода наименьших квадратов (FGLS, feasible generalized least squares) применительно к данной ситуации
\end{enumerate}
\end{enumerate}
Hint: Функция плотности одномерного нормального распределения имеет вид 
\[
f(x)=\frac{1}{\sqrt{2\pi}\sigma}\exp\left(-\frac{(x-\mu)^2}{2\sigma^2}  \right)
\]
% многомерное
%f(x)=(2\pi)^{-n/2} \det(\Omega)^{-1/2} \exp\left(-\frac{1}{2}(x-\mu)'\Omega^{-1}(x-\mu)\right)


\item В курсе теории вероятностей изучался тест о равенстве математических ожиданий по двум нормальным выборкам при предпосылке о равенстве дисперсий. Предложите состоятельный способ тестировать гипотезу о равенстве математических ожиданий без предпосылки равенства дисперсий.


\end{enumerate}




\end{document}