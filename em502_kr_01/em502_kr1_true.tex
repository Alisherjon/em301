\documentclass[pdftex,12pt,a4paper]{article}

% jan 2012


\usepackage[paper=a4paper,top=13.5mm, bottom=13.5mm,left=16.5mm,right=13.5mm,includefoot]{geometry}

\usepackage{etex} % расширение классического tex
% в частности позволяет подгружать гораздо больше пакетов, чем мы и займёмся далее


% sudo yum install texlive-bbm texlive-bbm-macros texlive-asymptote texlive-cm-super texlive-cyrillic texlive-pgfplots texlive-subfigure
% yum install texlive-chessboard texlive-skaknew % for \usepackage{chessboard}
% yum install texlive-minted texlive-navigator texlive-yax texlive-texapi

% растягиваем границы страницы
%\emergencystretch=2em \voffset=-2cm \hoffset=-1cm
%\unitlength=0.6mm \textwidth=17cm \textheight=25cm

\usepackage{makeidx} % для создания предметных указателей
\usepackage{verbatim} % для многострочных комментариев
%\usepackage{cmap} % для поиска русских слов в pdf --- устарело
%\usepackage[pdftex]{graphicx} % для вставки графики 
% omit pdftex option if not using pdflatex


%\usepackage{dsfont} % шрифт для единички с двойной палочкой (для индикатора события)
\usepackage{bbm} % шрифт - двойные буквы

\usepackage[colorlinks,hyperindex,unicode,breaklinks]{hyperref} % гиперссылки в pdf


\usepackage[utf8]{inputenc} % выбор кодировки файла
\usepackage[T2A]{fontenc} % кодировка шрифта
\usepackage[russian]{babel} % выбор языка

\usepackage{amssymb}
\usepackage{amsmath}
\usepackage{amsthm}
\usepackage{epsfig}
\usepackage{bm}
\usepackage{color}

\usepackage[usenames, dvipsnames, svgnames, table, rgb]{xcolor}
\usepackage{colortbl}

\usepackage{multicol}
\usepackage{multirow} % Слияние строк в таблице
\usepackage{dcolumn} % apsr table requirement

\usepackage{textcomp}  % Чтобы в формулах можно было русские буквы писать через \text{}

\usepackage{embedfile} % Чтобы код LaTeXа включился как приложение в PDF-файл

\usepackage{subfigure} % для создания нескольких рисунков внутри одного

\usepackage{tikz,pgfplots} % язык для рисования графики из latex'a
\usetikzlibrary{trees} % прибамбас в нем для рисовки деревьев
\usetikzlibrary{arrows} % прибамбас в нем для рисовки стрелочек подлиннее
\usepackage{tikz-qtree} % прибамбас в нем для рисовки деревьев
\pgfplotsset{compat=1.8}

%\usepackage{verse} % для стихов




\usepackage{enumitem}


\embedfile[desc={Исходный LaTeX файл}]{\jobname.tex} % Включение кода в выходной файл
\embedfile[desc={Стилевой файл}]{title_bor_utf8.tex}



% вместо горизонтальной делаем косую черточку в нестрогих неравенствах
\renewcommand{\le}{\leqslant}
\renewcommand{\ge}{\geqslant} 
\renewcommand{\leq}{\leqslant}
\renewcommand{\geq}{\geqslant}

% делаем короче интервал в списках 
\setlength{\itemsep}{0pt} 
\setlength{\parskip}{0pt} 
\setlength{\parsep}{0pt}

% свешиваем пунктуацию (т.е. знаки пунктуации могут вылезать за правую границу текста, при этом текст выглядит ровнее)
\usepackage{microtype}

% более красивые таблицы
\usepackage{booktabs}
% заповеди из докупентации: 
% 1. Не используйте вертикальные линни
% 2. Не используйте двойные линии
% 3. Единицы измерения - в шапку таблицы
% 4. Не сокращайте .1 вместо 0.1
% 5. Повторяющееся значение повторяйте, а не говорите "то же"


% DEFS
\def \mbf{\mathbf}
\def \msf{\mathsf}
\def \mbb{\mathbb}
\def \tbf{\textbf}
\def \tsf{\textsf}
\def \ttt{\texttt}
\def \tbb{\textbb}

\def \wh{\widehat}
\def \wt{\widetilde}
\def \ni{\noindent}
\def \ol{\overline}
\def \cd{\cdot}
\def \bl{\bigl}
\def \br{\bigr}
\def \Bl{\Bigl}
\def \Br{\Bigr}
\def \fr{\frac}
\def \bs{\backslash}
\def \lims{\limits}
\def \arg{{\operatorname{arg}}}
\def \dist{{\operatorname{dist}}}
\def \VC{{\operatorname{VCdim}}}
\def \card{{\operatorname{card}}}
\def \sgn{{\operatorname{sign}\,}}
\def \sign{{\operatorname{sign}\,}}
\def \xfs{(x_1,\ldots,x_{n-1})}
\def \Tr{{\operatorname{\mbf{Tr}}}}
\DeclareMathOperator*{\argmin}{arg\,min}
\DeclareMathOperator*{\argmax}{arg\,max}
\DeclareMathOperator*{\amn}{arg\,min}
\DeclareMathOperator*{\amx}{arg\,max}
\def \cov{{\operatorname{Cov}}}
\DeclareMathOperator{\Var}{Var}
\DeclareMathOperator{\Cov}{Cov}
\DeclareMathOperator{\Corr}{Corr}
\DeclareMathOperator{\trace}{tr}
%\DeclareMathOperator{\tr}{tr}
\DeclareMathOperator{\rank}{rank}
\DeclareMathOperator{\rk}{rank}


\def \xfs{(x_1,\ldots,x_{n-1})}
\def \ti{\tilde}
\def \wti{\widetilde}


\def \mL{\mathcal{L}}
\def \mW{\mathcal{W}}
\def \mH{\mathcal{H}}
\def \mC{\mathcal{C}}
\def \mE{\mathcal{E}}
\def \mN{\mathcal{N}}
\def \mA{\mathcal{A}}
\def \mB{\mathcal{B}}
\def \mU{\mathcal{U}}
\def \mV{\mathcal{V}}
\def \mF{\mathcal{F}}

\def \R{\mbb R}
\def \N{\mbb N}
\def \Z{\mbb Z}
\def \P{\mbb{P}}
%\def \p{\mbb{P}}
\def \E{\mbb{E}}
\def \D{\msf{D}}
\def \I{\mbf{I}}

\def \a{\alpha}
\def \b{\beta}
\def \t{\tau}
\def \dt{\delta}
\def \e{\varepsilon}
\def \ga{\gamma}
\def \kp{\varkappa}
\def \la{\lambda}
\def \sg{\sigma}
\def \sgm{\sigma}
\def \tt{\theta}
\def \ve{\varepsilon}
\def \Dt{\Delta}
\def \La{\Lambda}
\def \Sgm{\Sigma}
\def \Sg{\Sigma}
\def \Tt{\Theta}
\def \Om{\Omega}
\def \om{\omega}


\def \ni{\noindent}
\def \lq{\glqq}
\def \rq{\grqq}
\def \lbr{\linebreak}
\def \vsi{\vspace{0.1cm}}
\def \vsii{\vspace{0.2cm}}
\def \vsiii{\vspace{0.3cm}}
\def \vsiv{\vspace{0.4cm}}
\def \vsv{\vspace{0.5cm}}
\def \vsvi{\vspace{0.6cm}}
\def \vsvii{\vspace{0.7cm}}
\def \vsviii{\vspace{0.8cm}}
\def \vsix{\vspace{0.9cm}}
\def \VSI{\vspace{1cm}}
\def \VSII{\vspace{2cm}}
\def \VSIII{\vspace{3cm}}


\newcommand{\grad}{\mathrm{grad}}
\newcommand{\dx}[1]{\,\mathrm{d}#1} % для интеграла: маленький отступ и прямая d
\newcommand{\ind}[1]{\mathbbm{1}_{\{#1\}}} % Индикатор события
%\renewcommand{\to}{\rightarrow}
\newcommand{\eqdef}{\mathrel{\stackrel{\rm def}=}}
\newcommand{\iid}{\mathrel{\stackrel{\rm i.\,i.\,d.}\sim}}
\newcommand{\const}{\mathrm{const}}

%на всякий случай пока есть
%теоремы без нумерации и имени
%\newtheorem*{theor}{Теорема}

%"Определения","Замечания"
%и "Гипотезы" не нумеруются
\newtheorem*{definition}{Определение}
%\newtheorem*{rem}{Замечание}
%\newtheorem*{conj}{Гипотеза}

%"Теоремы" и "Леммы" нумеруются
%по главам и согласованно м/у собой
\newtheorem{theorem}{Теорема}
%\newtheorem{lemma}[theorem]{Лемма}

% Утверждения нумеруются по главам
% независимо от Лемм и Теорем
%\newtheorem{prop}{Утверждение}
%\newtheorem{cor}{Следствие}



% чисто эконометрические сокращения:
\def \hb{\hat{\beta}}
\def \hy{\hat{y}}
\def \hY{\hat{Y}}
\def \he{\hat{\varepsilon}}
\def \v1{\vec{1}}
\def \e{\varepsilon}
\def \hVar{\widehat{\Var}}
\def \hCorr{\widehat{\Corr}}


\begin{document}
\parindent=0 pt % отступ равен 0

\newcommand{\here}{\vspace{1pt}\begin{center}
\line(1,0){450}
\end{center}}

Контрольная работа 1 по эконометрике-2, 26.10.12.

Группа, ФИО\here


Поехали!!!

Короткие вопросы:

\begin{enumerate}

\item Методом наименьших квадратов оцените коэффициент $\beta$ в модели $y_i=\beta\cdot i +\varepsilon_i$.
\here

\item Априори известно, что парная регрессия должна проходить через точку $(0,1)$. Найдите мнк оценки в регрессии $\hy_i=\hb_1+\hb_2x_i$
\here


\item Аня и Настя утверждают, что лектор опоздал на 10 минут. Таня считает, что лектор опоздал на 3 минуты. С помощью мнк оцените на сколько опоздал лектор. 
\here

\item Эконометрист Вовочка оценил линейную регрессионную модель, где $y$ измерялся в тугриках. Затем он оценил ту же модель, но измерял $y$ в мунгу (1 тугрик = 100 мунгу). Как изменятся оценки коэффициентов?
\here

\item Какие из указанные моделей можно представить в линейном виде?
\begin{enumerate}
\item $y_i=\beta_1+\frac{\beta_2}{x_i}+\e_i$
\item $y_i=\exp(\beta_1+\beta_2 x_i+\e_i)$
\item $y_i=1+\frac{1}{\exp(\beta_1+\beta_2 x_i+\e_i)}$
\item $y_i=\frac{1}{1+\exp(\beta_1+\beta_2 x_i+\e_i)}$
\item $y_i=x_i^{\beta_2}e^{\beta_1+\e_i}$
\end{enumerate}
\here

\item Сформулируйте предпосылки теоремы Гаусса-Маркова
\here

\item Сформулируйте выводы теоремы Гаусса-Маркова
\here

\item Неограниченная регрессионная модель имеет вид $y_i=\beta_1+\b_2x_i+\b_3z_i+\b_4w_i+\e_i$. Какую регрессию следует оценить для проверки ограничений $\left\{\begin{array}{c}
\b_2=\b_3 \\
\b_4=0
\end{array}\right.$?
\here


\item Неограниченная регрессионная модель имеет вид $y_i=\beta_1+\b_2x_i+\b_3z_i+\b_4w_i+\e_i$. Какую регрессию следует оценить для проверки ограничений $\left\{\begin{array}{c}
\b_2+\b_3=1 \\
\b_3+\b_4=0
\end{array}\right.$?
\here

\item Если соответстующее p-значение равно $0.03$, то гипотеза $H_0: \beta=0$ будет отвергаться при уровнях значимости лежащих в интервале
\here

\item Числом обусловленности матрицы называется отношение
\here

\item Большие значения коэффициентов вздутия дисперсии, VIF, являются одним из признаков
\here

\end{enumerate}


Большие задачи:

\begin{enumerate}
\item Регрессионная модель  задана в матричном виде при помощи уравнения $y=X\beta+\varepsilon$, где $\beta=(\beta_1,\beta_2,\beta_3)'$.
Известно, что $\E(\varepsilon)=0$  и  $\Var(\varepsilon)=\sigma^2\cdot I$.
Известно также, что 

$y=\left(
\begin{array}{c} 
1\\ 
2\\ 
3\\ 
4\\ 
5
\end{array}\right)$, 
$X=\left(\begin{array}{ccc}
1 & 0 & 0 \\
1 & 0 & 0 \\
1 & 0 & 1 \\
1 & 1 & 0 \\
1 & 1 & 0 
\end{array}\right)$.


Для удобства расчетов приведены матрицы 


$X'X=\left(
\begin{array}{ccc} 
5 & 2 & 1\\ 
2 & 2 & 0\\ 
1 & 0 & 1 
\end{array}\right)$ и $(X'X)^{-1}=\frac{1}{2}\left(
\begin{array}{ccc} 
1 & -1 & -1 \\
-1 & 2 & 1 \\
-1 & 1 & 3
\end{array}\right)$.

\begin{enumerate}
\item Укажите число наблюдений.
\here
\item Укажите число регрессоров с учетом свободного члена.
\here
\item Рассчитайте при помощи метода наименьших квадратов $\hb$, оценку для вектора неизвестных коэффициентов.
\here
\item Рассчитайте $TSS=\sum (y_i-\bar{y})^2$, $RSS=\sum (y_i-\hat{y}_i)^2$ и $ESS=\sum (\hat{y}_i-\bar{y})^2$.
\here
\item Чему равен $\he_5$, МНК-остаток регрессии, соответствующий 5-ому наблюдению?
\here
\item Чему равен $R^2$  в модели? Прокомментируйте полученное значение с точки зрения качества оцененного уравнения регрессии.
\here
\item Рассчитайте несмещенную оценку для неизвестного параметра $\sigma^2$ регрессионной модели.
\here
\item Рассчитайте $\widehat{\Var}(\hb)$, оценку для ковариационной матрицы вектора МНК-коэффициентов $\hb$.  
\here
\item Найдите $\widehat{\Var}(\hb_1)$, несмещенную оценку дисперсии МНК-коэффициента $\hb_1$.
\here
\item Найдите $\widehat{\Cov}(\hb_1,\hb_2)$, несмещенную оценку ковариации МНК-коэффициентов $\hb_1$ и $\hb_2$.
\here
\item Найдите $\widehat{\Var}(\hb_1+\hb_2)$
\here
\item Найдите $\hCorr(\hb_1,\hb_2)$, оценку коэффициента корреляции МНК-коэффициентов $\hb_1$ и $\hb_2$.
\here
\item Найдите $s_{\hb_1}$, стандартную ошибку МНК-коэффициента $\hb_1$.
\here
\item Рассчитайте выборочную корреляцию $y$ и $\hy$.
\here

\end{enumerate}
\item Оценена модель зависимости продолжительности сна млекопитающих от массы мозга и массы тела:

\begin{verbatim}
lm(formula = sleep_total ~ brainwt + bodywt, data = msleep)

Coefficients:
              Estimate Std. Error t value Pr(>|t|)    
(Intercept) 10.6722485  0.5898643  18.093   <2e-16 ***
brainwt     -2.3518943  1.6180072  -1.454    0.152    
bodywt       0.0007953  0.0016703   0.476    0.636    
---
Signif. codes:  0 ‘***’ 0.001 ‘**’ 0.01 ‘*’ 0.05 ‘.’ 0.1 ‘ ’ 1 

Residual standard error: 4.193 on 53 degrees of freedom
  (27 observations deleted due to missingness)
Multiple R-squared: 0.1337,	Adjusted R-squared: 0.101 
F-statistic: 4.088 on 2 and 53 DF,  p-value: 0.02232 
\end{verbatim}


\begin{enumerate}
\item  Протестируйте на значимость регрессию <<в целом>> на уровне значимости 5\%
\begin{enumerate}
\item Сформулируйте основную и альтернативную гипотезу, которые соответствуют тесту
на значимость уравнения регрессии “в целом”.
\here
\item Приведите формулу для тестовой статистики.
\here
\item Укажите распределение тестовой статистики.
\here
\item Вычислите наблюдаемое значение тестовой статистики.
\here
\item Укажите границы области, где основная гипотеза не отвергается.
\here
\item Сделайте статистический вывод о значимости уравнения регрессии <<в целом>>
\here
\end{enumerate} 
\item Проверьте гипотезу $H_0: \beta_{bodywt} = 1$ против альтернативной гипотезы $H_a: \beta_{bodywt} < 1$. Уровень значимости 5\%.
\begin{enumerate}
\item Приведите формулу для тестовой статистики.
\here
\item Укажите распределение тестовой статистики.
\here
\item Вычислите наблюдаемое значение тестовой статистики.
\here
\item Укажите границы области, где основная гипотеза не отвергается.
\here
\item Сделайте статистический вывод.
\here
\end{enumerate}
\item Укажите оценки коэффициентов, значимые на 5\%-ом уровне значимости
\here
\end{enumerate}


\end{enumerate}



\end{document}