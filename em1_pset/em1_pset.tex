\documentclass[pdftex,12pt,a4paper]{article}

% jan 2012


\usepackage[paper=a4paper,top=13.5mm, bottom=13.5mm,left=16.5mm,right=13.5mm,includefoot]{geometry}

\usepackage{etex} % расширение классического tex
% в частности позволяет подгружать гораздо больше пакетов, чем мы и займёмся далее


% sudo yum install texlive-bbm texlive-bbm-macros texlive-asymptote texlive-cm-super texlive-cyrillic texlive-pgfplots texlive-subfigure
% yum install texlive-chessboard texlive-skaknew % for \usepackage{chessboard}
% yum install texlive-minted texlive-navigator texlive-yax texlive-texapi

% растягиваем границы страницы
%\emergencystretch=2em \voffset=-2cm \hoffset=-1cm
%\unitlength=0.6mm \textwidth=17cm \textheight=25cm

\usepackage{makeidx} % для создания предметных указателей
\usepackage{verbatim} % для многострочных комментариев
%\usepackage{cmap} % для поиска русских слов в pdf --- устарело
%\usepackage[pdftex]{graphicx} % для вставки графики 
% omit pdftex option if not using pdflatex


%\usepackage{dsfont} % шрифт для единички с двойной палочкой (для индикатора события)
\usepackage{bbm} % шрифт - двойные буквы

\usepackage[colorlinks,hyperindex,unicode,breaklinks]{hyperref} % гиперссылки в pdf


\usepackage[utf8]{inputenc} % выбор кодировки файла
\usepackage[T2A]{fontenc} % кодировка шрифта
\usepackage[russian]{babel} % выбор языка

\usepackage{amssymb}
\usepackage{amsmath}
\usepackage{amsthm}
\usepackage{epsfig}
\usepackage{bm}
\usepackage{color}

\usepackage[usenames, dvipsnames, svgnames, table, rgb]{xcolor}
\usepackage{colortbl}

\usepackage{multicol}
\usepackage{multirow} % Слияние строк в таблице
\usepackage{dcolumn} % apsr table requirement

\usepackage{textcomp}  % Чтобы в формулах можно было русские буквы писать через \text{}

\usepackage{embedfile} % Чтобы код LaTeXа включился как приложение в PDF-файл

\usepackage{subfigure} % для создания нескольких рисунков внутри одного

\usepackage{tikz,pgfplots} % язык для рисования графики из latex'a
\usetikzlibrary{trees} % прибамбас в нем для рисовки деревьев
\usetikzlibrary{arrows} % прибамбас в нем для рисовки стрелочек подлиннее
\usepackage{tikz-qtree} % прибамбас в нем для рисовки деревьев
\pgfplotsset{compat=1.8}

%\usepackage{verse} % для стихов




\usepackage{enumitem}


\embedfile[desc={Исходный LaTeX файл}]{\jobname.tex} % Включение кода в выходной файл
\embedfile[desc={Стилевой файл}]{title_bor_utf8.tex}



% вместо горизонтальной делаем косую черточку в нестрогих неравенствах
\renewcommand{\le}{\leqslant}
\renewcommand{\ge}{\geqslant} 
\renewcommand{\leq}{\leqslant}
\renewcommand{\geq}{\geqslant}

% делаем короче интервал в списках 
\setlength{\itemsep}{0pt} 
\setlength{\parskip}{0pt} 
\setlength{\parsep}{0pt}

% свешиваем пунктуацию (т.е. знаки пунктуации могут вылезать за правую границу текста, при этом текст выглядит ровнее)
\usepackage{microtype}

% более красивые таблицы
\usepackage{booktabs}
% заповеди из докупентации: 
% 1. Не используйте вертикальные линни
% 2. Не используйте двойные линии
% 3. Единицы измерения - в шапку таблицы
% 4. Не сокращайте .1 вместо 0.1
% 5. Повторяющееся значение повторяйте, а не говорите "то же"


% DEFS
\def \mbf{\mathbf}
\def \msf{\mathsf}
\def \mbb{\mathbb}
\def \tbf{\textbf}
\def \tsf{\textsf}
\def \ttt{\texttt}
\def \tbb{\textbb}

\def \wh{\widehat}
\def \wt{\widetilde}
\def \ni{\noindent}
\def \ol{\overline}
\def \cd{\cdot}
\def \bl{\bigl}
\def \br{\bigr}
\def \Bl{\Bigl}
\def \Br{\Bigr}
\def \fr{\frac}
\def \bs{\backslash}
\def \lims{\limits}
\def \arg{{\operatorname{arg}}}
\def \dist{{\operatorname{dist}}}
\def \VC{{\operatorname{VCdim}}}
\def \card{{\operatorname{card}}}
\def \sgn{{\operatorname{sign}\,}}
\def \sign{{\operatorname{sign}\,}}
\def \xfs{(x_1,\ldots,x_{n-1})}
\def \Tr{{\operatorname{\mbf{Tr}}}}
\DeclareMathOperator*{\argmin}{arg\,min}
\DeclareMathOperator*{\argmax}{arg\,max}
\DeclareMathOperator*{\amn}{arg\,min}
\DeclareMathOperator*{\amx}{arg\,max}
\def \cov{{\operatorname{Cov}}}
\DeclareMathOperator{\Var}{Var}
\DeclareMathOperator{\Cov}{Cov}
\DeclareMathOperator{\Corr}{Corr}
\DeclareMathOperator{\trace}{tr}
%\DeclareMathOperator{\tr}{tr}
\DeclareMathOperator{\rank}{rank}
\DeclareMathOperator{\rk}{rank}


\def \xfs{(x_1,\ldots,x_{n-1})}
\def \ti{\tilde}
\def \wti{\widetilde}


\def \mL{\mathcal{L}}
\def \mW{\mathcal{W}}
\def \mH{\mathcal{H}}
\def \mC{\mathcal{C}}
\def \mE{\mathcal{E}}
\def \mN{\mathcal{N}}
\def \mA{\mathcal{A}}
\def \mB{\mathcal{B}}
\def \mU{\mathcal{U}}
\def \mV{\mathcal{V}}
\def \mF{\mathcal{F}}

\def \R{\mbb R}
\def \N{\mbb N}
\def \Z{\mbb Z}
\def \P{\mbb{P}}
%\def \p{\mbb{P}}
\def \E{\mbb{E}}
\def \D{\msf{D}}
\def \I{\mbf{I}}

\def \a{\alpha}
\def \b{\beta}
\def \t{\tau}
\def \dt{\delta}
\def \e{\varepsilon}
\def \ga{\gamma}
\def \kp{\varkappa}
\def \la{\lambda}
\def \sg{\sigma}
\def \sgm{\sigma}
\def \tt{\theta}
\def \ve{\varepsilon}
\def \Dt{\Delta}
\def \La{\Lambda}
\def \Sgm{\Sigma}
\def \Sg{\Sigma}
\def \Tt{\Theta}
\def \Om{\Omega}
\def \om{\omega}


\def \ni{\noindent}
\def \lq{\glqq}
\def \rq{\grqq}
\def \lbr{\linebreak}
\def \vsi{\vspace{0.1cm}}
\def \vsii{\vspace{0.2cm}}
\def \vsiii{\vspace{0.3cm}}
\def \vsiv{\vspace{0.4cm}}
\def \vsv{\vspace{0.5cm}}
\def \vsvi{\vspace{0.6cm}}
\def \vsvii{\vspace{0.7cm}}
\def \vsviii{\vspace{0.8cm}}
\def \vsix{\vspace{0.9cm}}
\def \VSI{\vspace{1cm}}
\def \VSII{\vspace{2cm}}
\def \VSIII{\vspace{3cm}}


\newcommand{\grad}{\mathrm{grad}}
\newcommand{\dx}[1]{\,\mathrm{d}#1} % для интеграла: маленький отступ и прямая d
\newcommand{\ind}[1]{\mathbbm{1}_{\{#1\}}} % Индикатор события
%\renewcommand{\to}{\rightarrow}
\newcommand{\eqdef}{\mathrel{\stackrel{\rm def}=}}
\newcommand{\iid}{\mathrel{\stackrel{\rm i.\,i.\,d.}\sim}}
\newcommand{\const}{\mathrm{const}}

%на всякий случай пока есть
%теоремы без нумерации и имени
%\newtheorem*{theor}{Теорема}

%"Определения","Замечания"
%и "Гипотезы" не нумеруются
\newtheorem*{definition}{Определение}
%\newtheorem*{rem}{Замечание}
%\newtheorem*{conj}{Гипотеза}

%"Теоремы" и "Леммы" нумеруются
%по главам и согласованно м/у собой
\newtheorem{theorem}{Теорема}
%\newtheorem{lemma}[theorem]{Лемма}

% Утверждения нумеруются по главам
% независимо от Лемм и Теорем
%\newtheorem{prop}{Утверждение}
%\newtheorem{cor}{Следствие}


% чисто эконометрические сокращения:
\def \hb{\hat{\beta}}
\def \hy{\hat{y}}
\def \hY{\hat{Y}}
\def \he{\hat{\varepsilon}}

% временное решение
\newcommand{\solution}[1]{ {\tiny #1} }
\newcommand{\problem}[1]{#1}

\title{Задачник по эконометрике-1 \\ {\small (с шахматами и поэтэссами)}}
\author{Дмитрий Борзых, Борис Демешев}
\date{\today}

\makeindex % команда для создания предметного указателя
\bibliographystyle{plain} % стиль оформления ссылок


\begin{document}

\maketitle % печатаем заголовок


\parindent=0 pt % отступ равен 0



\begin{enumerate}
\item Регрессионная модель   задана в матричном виде при помощи уравнения $y=X\beta+\varepsilon$, где $\beta=$.
Известно, что $\E(\varepsilon)=0$  и  $\Var(\varepsilon)=\sigma^2\cdot I$.
Известно также, что $y=$, $X=$.
Для удобства расчетов ниже приведены матрицы 
 $X'X=$ и $(X'X)^{-1}=$.

\begin{enumerate}
\item Укажите число наблюдений.
\item Укажите число регрессоров с учетом свободного члена.
\item Рассчитайте $TSS=\sum (y_i-\bar{y})^2$, $RSS=\sum (y_i-\hat{y}_i)^2$ и $ESS=\sum (\hat{y}_i-\bar{y})^2$.
\item Рассчитайте при помощи метода наименьших квадратов $\hb$, оценку для вектора неизвестных коэффициентов.
\item Чему равен $\he_5$, МНК-остаток регрессии, соответствующий 5-ому наблюдению?
\item Чему равен $R^2$  в модели? Прокомментируйте полученное значение с точки зрения качества оцененного уравнения регрессии.
\item Используя приведенные выше данные, рассчитайте несмещенную оценку для неизвестного параметра $\sigma^2$ регрессионной модели.
\item Рассчитайте $\widehat{\Cov}(\hb)$, оценку для ковариационной матрицы вектора МНК-коэффициентов $\hb$.  
\item Найдите $\widehat{\Var}(\hb_1)$, несмещенную оценку дисперсии МНК-коэффициента $\hb_1$.
\item Найдите $\widehat{\Var}(\hb_2)$, несмещенную оценку дисперсии МНК-коэффициента $\hb_2$.
\item Найдите $\widehat{\Cov}(\hb_1,\hb_2)$, несмещенную оценку ковариации МНК-коэффициентов $\hb_1$ и $\hb_2$.
\item Найдите $\widehat{\Var}(\hb_1+\hb_2)$, $\widehat{\Var}(\hb_1-\hb_2)$, $\widehat{\Var}(\hb_1+\hb_2+\hb_3)$, $\widehat{\Var}(\hb_1+\hb_2-2\hb_3)$
\item Найдите $\Corr(\hb_1,\hb_2)$, оценку коэффициента корреляции МНК-коэффициентов $\hb_1$ и $\hb_2$.
\item Найдите $s_{\hb_1}$, стандартную ошибку МНК-коэффициента $\hb_1$.
\end{enumerate}

\item Априори известно, что парная регрессия должна проходить через точку $(x_{0},y_{0})$.
\begin{enumerate}
\item  Выведите формулы МНК оценок;
\item В предположениях теоремы Гаусса-Маркова найдите дисперсии и средние оценок 
\end{enumerate}

\solution{Вроде бы равносильно переносу начала координат и применению результата для регрессии без свободного члена. Должна остаться несмещенность. }


\item \problem{ Перед нами два золотых слитка и весы, производящие взвешивания с ошибками. Взвесив первый слиток, мы получили результат $300$ грамм, взвесив второй слиток --- $200$ грамм, взвесив оба слитка --- $400$ грамм. Оцените вес каждого слитка методом наименьших квадратов.}
\solution{ $(300-\hb_1)^2+(200-\hb_2)^2+(400-\hb_1-\hb_2)^2\to\min$ }


\item \problem{ Слитки-вариант. Перед нами два золотых слитка и весы, производящие взвешивания с ошибками. Взвесив первый слиток, мы получили результат $300$ грамм, взвесив второй слиток --- $200$ грамм, взвесив оба слитка --- $400$ грамм. Предположим, что ошибки взвешивания --- независимые одинаково распределенные случайные величины с нулевым средним. 
\begin{enumerate}
\item Найдите несмещеную оценку веса первого шара, обладающую наименьшей дисперсией.
\item Как можно проинтерпретировать нулевое математическое ожидание ошибки взвешивания? 
\end{enumerate} }
\solution{ Как отсутствие систематической ошибки.} 


\item \problem{Даны $n$ пар чисел: $(x_1, y_1)$, \ldots, $(x_n,y_n)$. Мы прогнозируем $y_i$ по формуле $\hy_i=\hb x_i$. Найдите $\hb$ методом наименьших квадратов. }
\solution{$\hb=\sum x_i y_i/\sum x_i^2$}

\item \problem{Даны $n$ чисел: $y_1$, \ldots, $y_n$. Мы прогнозируем $y_i$ по формуле $\hy_i=\hb$. Найдите $\hb$ методом наименьших квадратов. }
\solution{$\hb=\bar{y}$}

\item Регрессия на дамми-переменную...


\end{enumerate}

\end{document}