\documentclass[pdftex,12pt,a4paper]{article}

% размер листа бумаги
\usepackage[paper=a4paper,top=13.5mm, bottom=13.5mm,left=16.5mm,right=13.5mm,includefoot]{geometry}

\usepackage{makeidx} % для создания предметного указателя 
\usepackage{cmap} % поиск русских букв в pdf 

\usepackage[pdftex]{graphicx} % вставка картинок
\usepackage[colorlinks,hyperindex,unicode]{hyperref} % гиперссылки

\usepackage[utf8]{inputenc} % кодировка файла
\usepackage[T2A]{fontenc} % кодировка шрифтов
\usepackage[russian]{babel} % русификация

\usepackage{amssymb}
\usepackage{amsmath}
\usepackage{amsthm}
\usepackage{epsfig}
\usepackage{bm}
\usepackage{color}
\usepackage{multicol}

\usepackage{textcomp}  % Чтобы в формулах можно было русские буквы писать через \text{}
\usepackage{enumitem} % дополнительные плюшки для списков
%  например \begin{enumerate}[resume] позволяет продолжить нумерацию в новом списке

\title{Pale Blue Dot}
\author{5 сентября 2013 года}
\date{}

\begin{document}


\pagenumbering{gobble}

\begin{titlepage}

\maketitle
%\parindent=0 pt % no indent

{\footnotesize  Тридцать шесть лет назад, 5 сентября 1977 года, был запущен искусственный спутник Voyager-1. Сейчас он находится в 18 миллиардах километрах от Земли. Это огромное расстояние трудно представить. Радиосигнал проходит его примерно за 18 часов. Когда Voyager-1 был в шести миллиардах километрах от Земли, он сделал свою самую известную фотографию, Pale Blue Dot. }

\bigskip 

\begin{quotation}
Взгляните ещё раз на эту точку. Это здесь. Это наш дом. Это мы. Все, кого вы любите, все, кого вы знаете, все, о ком вы когда-либо слышали, все когда-либо существовавшие люди прожили свои жизни на ней. Множество наших наслаждений и страданий, тысячи самоуверенных религий, идеологий и экономических доктрин, каждый охотник и собиратель, каждый герой и трус, каждый созидатель и разрушитель цивилизаций, каждый король и крестьянин, каждая влюблённая пара, каждая мать и каждый отец, каждый способный ребёнок, изобретатель и путешественник, каждый преподаватель этики, каждый лживый политик, каждая «суперзвезда», каждый «величайший лидер», каждый святой и грешник в истории нашего вида жили здесь — на соринке, подвешенной в солнечном луче.

Земля — очень маленькая сцена на безбрежной космической арене. Подумайте о реках крови, пролитых всеми этими генералами и императорами, чтобы, в лучах славы и триумфа, они могли стать кратковременными хозяевами части песчинки. Подумайте о бесконечных жестокостях, совершаемых обитателями одного уголка этой точки над едва отличимыми обитателями другого уголка. О том, как часты меж ними разногласия, о том, как жаждут они убивать друг друга, о том, как горяча их ненависть.

Наши позёрства, наша воображаемая значимость, иллюзия о нашем привилегированном статусе во вселенной — все они пасуют перед этой точкой бледного света. Наша планета — лишь одинокая пылинка в окружающей космической тьме. В этой грандиозной пустоте нет ни намёка на то, что кто-то придёт нам на помощь, дабы спасти нас от нашего же невежества.

Земля — пока единственный известный мир, способный поддерживать жизнь. Нам больше некуда уйти — по крайней мере, в ближайшем будущем. Побывать — да. Колонизировать — ещё нет. Нравится вам это или нет — Земля сейчас наш дом.

Говорят, астрономия прививает скромность и укрепляет характер. Наверное, нет лучшей демонстрации глупого человеческого зазнайства, чем эта отстранённая картина нашего крошечного мира. Мне кажется, она подчёркивает нашу ответственность, наш долг быть добрее друг с другом, дорожить и лелеять бледно-голубую точку — наш единственный дом.
\end{quotation}
Карл Саган

\end{titlepage}


\end{document}

