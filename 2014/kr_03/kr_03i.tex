\documentclass[12pt,a4paper]{article}
\usepackage[utf8]{inputenc}
\usepackage[russian]{babel}

\usepackage{amsmath}
\usepackage{amsfonts}
\usepackage{amssymb}
\usepackage[left=2cm,right=2cm,top=2cm,bottom=2cm]{geometry}

\DeclareMathOperator{\Var}{Var}
\DeclareMathOperator{\Cov}{Cov}
\DeclareMathOperator{\plim}{plim}

\newcommand{\e}{\varepsilon}
\newcommand{\E}{\mathbb{E}}
\renewcommand{\P}{\mathbb{P}}
\begin{document}

Праздник по эконометрике номер 3 

Примечание: во всех задачах, если явно не сказано обратное, предполагается, что выполнены стандартные предпосылки классической линейной регрессионной модели. 

\begin{enumerate}

\item  Рассмотрим следующую модель зависимости почасовой оплаты труда $W$ от уровня образования $Educ$, возраста $Age$, уровня образования родителей $Fathedu$ и $Mothedu$:
\[
\widehat{\ln W} = \hat{\beta}_1 + \hat{\beta}_2 Educ + \hat{\beta}_3 Age + \hat{\beta}_4 Age^2+ \hat{\beta}_5 Fathedu + \hat{\beta}_6 Mothedu
\]
\[
R^2 = 0.341, n = 27
\]
\begin{enumerate}
\item Напишите спецификацию регрессии с ограничениями для проверки статистической гипотезы $H_0: \beta_5 = 2\beta_4$
\item Дайте интерпретацию проверяемой гипотезе
\item Для регрессии с ограничением был вычислен коэффициент $R_{R}^2 = 0.296$. На уровне значимости $5\%$ проверьте нулевую гипотезу
\end{enumerate}



\item По ежегодным данным с 2002 по 2009 год оценивался тренд в динамике общей стоимости экспорта из РФ: $Exp_t=\beta_1+\beta_2t+\varepsilon_t$, где t --- год ($t=0$ для 2002 г., $t=1$ для 2003 г., \ldots, $t=7$
для 2009 г.), $Exp_t$ --- стоимость экспорта из РФ во все страны в млрд. долл. Оценённое уравнение  выглядит так: $\widehat{Exp}_t=111.9+43.2t$. Получены также оценки дисперсии случайной ошибки $\hat{\sigma}^2=4009$ и ковариационной матрицы оценок коэффициентов:
\[
\widehat{Var}(\hat{\beta})=\begin{pmatrix}
1671 & -334 \\
-334 & 95
\end{pmatrix}
\]

\begin{enumerate}
\item Постройте 95\%-ый доверительный интервал для коэффициента $\beta_2$
\item Спрогнозируйте стоимость экспорта на 2010 год и постройте 90\%-ый предиктивный интервал для прогноза.
\end{enumerate}
\item Имеется $100$ наблюдений. Исследователь Вениамин предполагает, что дисперсия случайной ошибки в последних $50$-ти наблюдениях в 4 раза выше, чем в первых $50$-ти, в частности $\Var(\varepsilon_1)=\sigma^2$, а $\Var(\varepsilon_{100})=4\sigma^2$. Вениамин оценивает модель $y_i=\beta x_i +\varepsilon_i$ с помощью МНК.
\begin{enumerate}
\item Найдите истинную дисперсию МНК оценки коэффициента $\beta$
\item Предложите более эффективную оценку $\hat{\beta}^{alt}$
\item Чему равна истинная дисперсия новой оценки?
\item Подробно опишите любой способ, который позволяет протестировать гипотезу о гомоскедастичности против предположения Вениамина о дисперсии.
\end{enumerate}
\newpage

\item Закон больших чисел гласит, что если $z_i$ независимы и одинаково распределены, то $\plim \bar{z}_n = \E(z_1)$. Предположим, что регрессоры --- стохастические, а именно, наблюдения являются случайной выборкой (то есть отдельные наблюдения независимы и одинаково распределены), и  $\E(\varepsilon | X)=0$. Модель имеет вид:

\[
y_i=\beta_1 + \beta_2 x_i +\beta_3 w_i +\varepsilon_i
\]

\begin{enumerate}
\item Найдите $\E(\varepsilon)$, $\E(x_1 \cdot \varepsilon_1)$
\item Найдите $\plim \frac{1}{n}X'\varepsilon$
\item Найдите $\plim \frac{1}{n}X'X$
\item Докажите, что вектор МНК оценок $\hat{\beta}$ является состоятельным
\end{enumerate}


\item Эконометресса Эвридика хочет оценить модель $y_i=\beta_1 + \beta_2 x_i +\beta_3 z_i + \e_i$. К сожалению, она измеряет зависимую переменную с ошибкой. Т.е. вместо $y_i$ она знает значение $y_i^*=y_i+u_i$ и использует его в качестве зависимой переменной при оценке регрессии. Ошибки измерения $u_i$ некоррелированы между собой и с $\e_i$, имеют нулевое математическое ожидание и постоянную дисперсию $\sigma^2_u$.
\begin{enumerate}
\item Будут ли оценки Эвридики несмещенными?
\item Могут ли дисперсии оценок Эвридики быть ниже чем дисперсии МНК оценок при использовании настоящего $y_i$?
\item Могут ли оценки дисперсий оценок Эвридики быть ниже чем оценок дисперсий МНК оценок при использовании настоящего $y_i$?
\end{enumerate}


\end{enumerate}



\end{document}