\documentclass[12pt,a4paper]{article}
\usepackage[utf8]{inputenc}
\usepackage[russian]{babel}

\usepackage{amsmath}
\usepackage{amsfonts}
\usepackage{amssymb}
\usepackage{graphicx}
\usepackage[left=2cm,right=2cm,top=2cm,bottom=2cm]{geometry}
\begin{document}

Эконометрика. 25 марта 2014.

\begin{enumerate}
\item  Случайные величины $X_{1}$, ..., $X_{n}$ --- независимы и одинаково распределены с функцией плотности $f(x)=\frac{\theta \cdot\left(\ln x\right)^{\theta -1}}{x} $  при  $x\in
\left[1;e\right]$. По выборке из 100 наблюдений оказалось, что $\sum{\ln(\ln(X_{i}))}=-30$ 
\begin{enumerate}
\item Найдите ML оценку параметра $\theta$
\item Постройте 95\% доверительный интервал для $\theta$
\item С помощью теста отношения правдоподобия и теста Вальда проверьте гипотезу о том, что $\theta=1$ на уровне значимости 5\%. 
\end{enumerate}


\item Рассмотрим набор данных по средней стоимости жилья в окрестностях Бостона. Данные можно загрузить в табличку \verb|b| командой \verb|b <- Boston|. Краткую справку о наборе данных можно получить командой \verb|help(Boston)|.
\begin{enumerate}
\item Постройте регрессию средней стоимости жилья на доступность образования, \verb|ptratio|, среднее количеcтво комнат в доме, удаленность от мест работы, \verb|dis|, и уровень преступности. Выпишите полученное уравнение регрессии
\item Какие коэффициенты значимы?
\item Постройте доверительный интервал для коэффициента при \verb|ptratio| без учета гетероскедастичности.
\item Проведите тест Уайта на гетероскедастичность. Каков результат теста?
\item Предположим, что дисперсия цены монотонно связана с возрастом домов. Проведите тест Голдфельда-Квандта, отобрав по 200 наблюдений с высокой и 200 наблюдений с низкой дисперсией. Каков результат теста?
\item Проверьте гипотезы о значимости коэффициентов используя стандартные ошибки устойчивые к гетероскедастичности.
\item Постройте доверительный интервал для коэффициента при \verb|ptratio| с учетом гетероскедастичности.
\end{enumerate}

\item Рассмотрим набор данных по психологическим характеристикам индивидов влияющим на желание участвовать волонтером в исследованиях. Данные можно загрузить в табличку \verb|d| командой \verb|d <- Cowles|. Краткую справку о наборе данных можно получить командой \verb|help(Cowles)|.

\begin{enumerate}
\item Оцените логит-модель, взяв в качестве зависимой переменной волонтерское участие в исследованиях. Выпишите уравнение регрессии.
\item Какие коэффициенты значимы?
\item Рассчитайте предельные эффекты для среднестатистического индивида. На сколько изменяется  вероятность участвовать волонтером при росте индекса нервозности, \verb|neuroticism|, на единицу?
\item Постройте прогноз вероятности быть волонтером вместе с 95\%-ым доверительным интервалом для девушки с индексом нервозности равным 15 и индексом экстравертности равным 20.
\end{enumerate}



\end{enumerate}


\end{document}