\documentclass[a4paper]{article}
\usepackage[utf8]{inputenc}
\usepackage[russian]{babel}

\usepackage{amsmath}
\usepackage{amsfonts}
\usepackage{amssymb}
\usepackage[paper=a4paper,top=20mm, bottom=10mm,left=15mm,right=15mm,includefoot]{geometry}
\usepackage[pdftex,unicode,colorlinks=true,urlcolor=blue,hyperindex,breaklinks]{hyperref} 
\begin{document}



\pagestyle{empty}
\section*{Домашка <<Титаник>>}


\begin{enumerate}

\item Зарегистрируйтесь на сайте \url{www.kaggle.com}  в конкурсе <<Titanic: Machine Learning from Disaster>>. В работе укажите login, использованный при регистрации.

\item Проанализируйте данные графически и с помощью описательных статистик (среднее, мода, медиана и т.д.)

Прокомментируйте графики, обратите внимание на количество пропущенных значений.

\item Оцените logit и probit модели.

Приведите оценки моделей. Какие коэффициенты значимы? Прокомментируйте знак коэффициентов. Посчитайте и сравните предельные эффекты.

\item Оцените random forest и SVM модели. 

Параметры методов подберите с помощью кросс-валидации. Можно применять любые другие подходы, не только random forest и SVM. Другой подход следует описать в тексте.


\item <<Если бы я был пассажиром Титаника, то я спасся бы с вероятностью\ldots>>. 

С помощью логит и пробит моделей постройте 95\%-ый доверительный интервал для вероятности своего спасения. Для random forest --- только точечный прогноз вероятности, для svm --- только прогноз типа <<да>>/<<нет>>.


\item Подумайте, чем можно заполнить пропущенные значения. Заполните пропущенные значения и заново оцените logit, random forest и svm. Насколько сильно меняется качество оцененных моделей?


\item Сравните все использованные подходы по прогнозной силе на тестовой выборке с сайта. Какой оказался наилучшим?

\item При прогнозировании и расчете предельных эффектов используйте свои фактические пол и возраст, а остальные объясняющие переменные --- выбирайте согласно своей фантазии :)

\item Срок сдачи --- 30 апреля 2014 года. 

Работа принимается исключительно в печатном виде с применением грамотного программирования R + \LaTeX. Каждый день более поздней сдачи умножает оценку за работу на $0.8$.  Работа должна представлять слитный текст, код скрывать не нужно. В конце должна быть команда \verb|sessionInfo()|.

\item Популярные ошибки прошлой домашки будут караться со всей строгостью военного времени! 

Список популярных ошибок, \url{}. Цикл заметок про R, \url{}.
 
\end{enumerate}





\end{document}