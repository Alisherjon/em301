\documentclass[pdftex,12pt,a4paper]{article}

% jan 2012


\usepackage[paper=a4paper,top=13.5mm, bottom=13.5mm,left=16.5mm,right=13.5mm,includefoot]{geometry}

\usepackage{etex} % расширение классического tex
% в частности позволяет подгружать гораздо больше пакетов, чем мы и займёмся далее


% sudo yum install texlive-bbm texlive-bbm-macros texlive-asymptote texlive-cm-super texlive-cyrillic texlive-pgfplots texlive-subfigure
% yum install texlive-chessboard texlive-skaknew % for \usepackage{chessboard}
% yum install texlive-minted texlive-navigator texlive-yax texlive-texapi

% растягиваем границы страницы
%\emergencystretch=2em \voffset=-2cm \hoffset=-1cm
%\unitlength=0.6mm \textwidth=17cm \textheight=25cm

\usepackage{makeidx} % для создания предметных указателей
\usepackage{verbatim} % для многострочных комментариев
%\usepackage{cmap} % для поиска русских слов в pdf --- устарело
%\usepackage[pdftex]{graphicx} % для вставки графики 
% omit pdftex option if not using pdflatex


%\usepackage{dsfont} % шрифт для единички с двойной палочкой (для индикатора события)
\usepackage{bbm} % шрифт - двойные буквы

\usepackage[colorlinks,hyperindex,unicode,breaklinks]{hyperref} % гиперссылки в pdf


\usepackage[utf8]{inputenc} % выбор кодировки файла
\usepackage[T2A]{fontenc} % кодировка шрифта
\usepackage[russian]{babel} % выбор языка

\usepackage{amssymb}
\usepackage{amsmath}
\usepackage{amsthm}
\usepackage{epsfig}
\usepackage{bm}
\usepackage{color}

\usepackage[usenames, dvipsnames, svgnames, table, rgb]{xcolor}
\usepackage{colortbl}

\usepackage{multicol}
\usepackage{multirow} % Слияние строк в таблице
\usepackage{dcolumn} % apsr table requirement

\usepackage{textcomp}  % Чтобы в формулах можно было русские буквы писать через \text{}

\usepackage{embedfile} % Чтобы код LaTeXа включился как приложение в PDF-файл

\usepackage{subfigure} % для создания нескольких рисунков внутри одного

\usepackage{tikz,pgfplots} % язык для рисования графики из latex'a
\usetikzlibrary{trees} % прибамбас в нем для рисовки деревьев
\usetikzlibrary{arrows} % прибамбас в нем для рисовки стрелочек подлиннее
\usepackage{tikz-qtree} % прибамбас в нем для рисовки деревьев
\pgfplotsset{compat=1.8}

%\usepackage{verse} % для стихов




\usepackage{enumitem}


\embedfile[desc={Исходный LaTeX файл}]{\jobname.tex} % Включение кода в выходной файл
\embedfile[desc={Стилевой файл}]{title_bor_utf8.tex}



% вместо горизонтальной делаем косую черточку в нестрогих неравенствах
\renewcommand{\le}{\leqslant}
\renewcommand{\ge}{\geqslant} 
\renewcommand{\leq}{\leqslant}
\renewcommand{\geq}{\geqslant}

% делаем короче интервал в списках 
\setlength{\itemsep}{0pt} 
\setlength{\parskip}{0pt} 
\setlength{\parsep}{0pt}

% свешиваем пунктуацию (т.е. знаки пунктуации могут вылезать за правую границу текста, при этом текст выглядит ровнее)
\usepackage{microtype}

% более красивые таблицы
\usepackage{booktabs}
% заповеди из докупентации: 
% 1. Не используйте вертикальные линни
% 2. Не используйте двойные линии
% 3. Единицы измерения - в шапку таблицы
% 4. Не сокращайте .1 вместо 0.1
% 5. Повторяющееся значение повторяйте, а не говорите "то же"


% DEFS
\def \mbf{\mathbf}
\def \msf{\mathsf}
\def \mbb{\mathbb}
\def \tbf{\textbf}
\def \tsf{\textsf}
\def \ttt{\texttt}
\def \tbb{\textbb}

\def \wh{\widehat}
\def \wt{\widetilde}
\def \ni{\noindent}
\def \ol{\overline}
\def \cd{\cdot}
\def \bl{\bigl}
\def \br{\bigr}
\def \Bl{\Bigl}
\def \Br{\Bigr}
\def \fr{\frac}
\def \bs{\backslash}
\def \lims{\limits}
\def \arg{{\operatorname{arg}}}
\def \dist{{\operatorname{dist}}}
\def \VC{{\operatorname{VCdim}}}
\def \card{{\operatorname{card}}}
\def \sgn{{\operatorname{sign}\,}}
\def \sign{{\operatorname{sign}\,}}
\def \xfs{(x_1,\ldots,x_{n-1})}
\def \Tr{{\operatorname{\mbf{Tr}}}}
\DeclareMathOperator*{\argmin}{arg\,min}
\DeclareMathOperator*{\argmax}{arg\,max}
\DeclareMathOperator*{\amn}{arg\,min}
\DeclareMathOperator*{\amx}{arg\,max}
\def \cov{{\operatorname{Cov}}}
\DeclareMathOperator{\Var}{Var}
\DeclareMathOperator{\Cov}{Cov}
\DeclareMathOperator{\Corr}{Corr}
\DeclareMathOperator{\trace}{tr}
%\DeclareMathOperator{\tr}{tr}
\DeclareMathOperator{\rank}{rank}
\DeclareMathOperator{\rk}{rank}


\def \xfs{(x_1,\ldots,x_{n-1})}
\def \ti{\tilde}
\def \wti{\widetilde}


\def \mL{\mathcal{L}}
\def \mW{\mathcal{W}}
\def \mH{\mathcal{H}}
\def \mC{\mathcal{C}}
\def \mE{\mathcal{E}}
\def \mN{\mathcal{N}}
\def \mA{\mathcal{A}}
\def \mB{\mathcal{B}}
\def \mU{\mathcal{U}}
\def \mV{\mathcal{V}}
\def \mF{\mathcal{F}}

\def \R{\mbb R}
\def \N{\mbb N}
\def \Z{\mbb Z}
\def \P{\mbb{P}}
%\def \p{\mbb{P}}
\def \E{\mbb{E}}
\def \D{\msf{D}}
\def \I{\mbf{I}}

\def \a{\alpha}
\def \b{\beta}
\def \t{\tau}
\def \dt{\delta}
\def \e{\varepsilon}
\def \ga{\gamma}
\def \kp{\varkappa}
\def \la{\lambda}
\def \sg{\sigma}
\def \sgm{\sigma}
\def \tt{\theta}
\def \ve{\varepsilon}
\def \Dt{\Delta}
\def \La{\Lambda}
\def \Sgm{\Sigma}
\def \Sg{\Sigma}
\def \Tt{\Theta}
\def \Om{\Omega}
\def \om{\omega}


\def \ni{\noindent}
\def \lq{\glqq}
\def \rq{\grqq}
\def \lbr{\linebreak}
\def \vsi{\vspace{0.1cm}}
\def \vsii{\vspace{0.2cm}}
\def \vsiii{\vspace{0.3cm}}
\def \vsiv{\vspace{0.4cm}}
\def \vsv{\vspace{0.5cm}}
\def \vsvi{\vspace{0.6cm}}
\def \vsvii{\vspace{0.7cm}}
\def \vsviii{\vspace{0.8cm}}
\def \vsix{\vspace{0.9cm}}
\def \VSI{\vspace{1cm}}
\def \VSII{\vspace{2cm}}
\def \VSIII{\vspace{3cm}}


\newcommand{\grad}{\mathrm{grad}}
\newcommand{\dx}[1]{\,\mathrm{d}#1} % для интеграла: маленький отступ и прямая d
\newcommand{\ind}[1]{\mathbbm{1}_{\{#1\}}} % Индикатор события
%\renewcommand{\to}{\rightarrow}
\newcommand{\eqdef}{\mathrel{\stackrel{\rm def}=}}
\newcommand{\iid}{\mathrel{\stackrel{\rm i.\,i.\,d.}\sim}}
\newcommand{\const}{\mathrm{const}}

%на всякий случай пока есть
%теоремы без нумерации и имени
%\newtheorem*{theor}{Теорема}

%"Определения","Замечания"
%и "Гипотезы" не нумеруются
\newtheorem*{definition}{Определение}
%\newtheorem*{rem}{Замечание}
%\newtheorem*{conj}{Гипотеза}

%"Теоремы" и "Леммы" нумеруются
%по главам и согласованно м/у собой
\newtheorem{theorem}{Теорема}
%\newtheorem{lemma}[theorem]{Лемма}

% Утверждения нумеруются по главам
% независимо от Лемм и Теорем
%\newtheorem{prop}{Утверждение}
%\newtheorem{cor}{Следствие}


% чисто эконометрические сокращения:
\def \hb{\hat{\beta}}
\def \hy{\hat{y}}
\def \hY{\hat{Y}}
\def \he{\hat{\varepsilon}}
\def \hCorr{\widehat{\Corr}}
\def \hVar{\widehat{\Var}}
\def \hCov{\widehat{\Cov}}


\begin{document}
\parindent=0 pt % отступ равен 0

\listoftodos

\section{Конвенция}


$y$ --- вектор столбец зависимых переменных, наблюдаемый случайный

$\beta$ --- вектор столбец неизвестных параметров, ненаблюдаемый, неслучайный

$\hy$ --- прогноз $y$ полученный по некоторой модели, наблюдаемый, случайный

$\hb$ --- оценки $\beta$

$X$ --- матрица всех объясняющих переменных

$\e$

$\he$


Некоторые авторы используют обозначения:

$Y$ и $y$ для разных вещей, $y=Y-\bar{Y}$.




\section{Семинар 1}

Неформальное определение. Если матрица $A$ квадратная, то её определителем называется площадь/объём параллелограмма/параллелепипеда образованного векторами-столбцами матрицы. Знак определителя задаётся порядком следования векторов. 


Свойства определителя:
\begin{enumerate}
\item $\det(AB)=\det(A)\det(B)=\det(BA)$, если $A$ и $B$ квадратные
\item $\det(A)=\prod \lambda_i$
\end{enumerate}


Определение. Если матрица $A$ квадратная, то её следом называется сумма диагональных элементов, $\trace(A)=\sum a_{ii}$.


Свойства следа:
\begin{enumerate}
\item $\trace(AB)=\trace(BA)$, если $AB$ и $BA$ существуют. При этом $A$ и $B$ могут не быть квадратными матрицами.
\item $\trace(A)=\sum \lambda_i$
\end{enumerate}


Добавить про геометрический смысл следа, \url{http://mathoverflow.net/questions/13526/geometric-interpretation-of-trace}.


Определение. Вектор $x$ называется собственным вектором матрицы $A$, если при умножении на матрицу $A$ он остается на той же прямой, т.е. $Ax=\lambda x$


Определение. Число $\lambda$ называется собственным числом матрицы $A$, если есть вектор $x$, который при умножении на матрицу $A$ изменяется в $\lambda$ раз, т.е. $Ax=\lambda x$.




Метод наименьших квадратов (МНК), ordinary least squares (OLS):


Есть $n$ наблюдений, $y_1$, ..., $y_n$. Есть модель, которая даёт прогнозы, $\hat{y}_1$, ..., $\hat{y}_n$. Эта модель зависит от вектора неизвестных параметров, $\beta$. МНК предлагает в качестве оценок неизвестных параметров взять такое $\hb$, чтобы минимизировать $\sum (y_i-\hat{y}_i)^2$.

\section{Семинар 2}

Контрольная-1

\section{Картинка}


Утверждение. $\sCorr^2(y,\hy)=R^2$

Доказательство. По определению, $\sCorr(y,\hy)=\frac{(y-\bar{y})(\hy-\bar{\hy})}{|y-\bar{y}||\hy-\bar{\hy}|}$. Поскольку в регрессии присутствует свободный член, $\bar{\hy}=\bar{y}$. Значит, 
\begin{equation}
\sCorr(y,\hy)=\frac{(y-\bar{y})(\hy-\bar{y})}{|y-\bar{y}||\hy-\bar{y}|}=\cos(y-\bar{y},\hy-\bar{y})
\end{equation}
По определению, $R^2=\frac{|\hy-\bar{y}|^2}{|y-\bar{y}|^2}=\cos^2(y-\bar{y},\hy-\bar{y})$



Опыт: лучший результат у меня получается с обозначением $(\bar{y},\ldots,\bar{y})$.


\section{Мегаматрица}
 
\begin{theorem}
След и математическое ожидание можно переставлять, $\E(\tr(A))=\tr(\E(A))$.
\end{theorem} 

\begin{theorem}
Математическое ожидание квадратичной формы
\begin{equation}
\E(x'Ax)=\tr(A\Var(x))+\E(x')A\E(x)
\end{equation}
\end{theorem}
\begin{proof}
Мы будем пользоваться простым приёмом. Если $u$ --- это скаляр, вектор размера 1 на 1, то $\tr(u)=u$.

Поехали,
\begin{equation}
\E(x'Ax)=\E(\tr(x'Ax))=\E(\tr(Axx'))=\tr(\E(Axx'))=\tr(A\E(xx'))
\end{equation}

По определению дисперсии, $\Var(x)=\E(xx')-\E(x)\E(x')$. Поэтому:
\begin{equation}
\tr(A\E(xx'))=\tr(A(\Var(x)+\E(x)\E(x')))=\tr(A\Var(x))+\tr(A\E(x)\E(x'))
\end{equation}

И готовимся снова использовать приём $\tr(u)=u$:
\begin{equation}
\tr(A\Var(x))+\tr(A\E(x)\E(x'))=\tr(A\Var(x))+\tr(\E(x')A\E(x))=
\tr(A\Var(x))+\E(x')A\E(x)
\end{equation}

\end{proof}


\section{Парадигма Случайных величин}

В парадигме случайных величин накладывают разные предпосылки.


Пример 1. Ошибки измерения в регрессорах


Пример 2. Независимые наблюдения


Пример 3. Стационарный процесс




Обозначим $X_{i.}$ --- $i$-ая строка матрицы $X$.

Вариант 0.

\begin{enumerate}
\item Регрессоры $X_{i.}$, относящиеся к разным $i$ некоррелированы.
\item Ковариационная матрица $X_{i.}$ не зависит от $i$.
\item Зависимая переменная представима в виде $y_i=X_{i.}\beta+\e_i$
\item Величины $\e_i$ некоррелированы, $\E(\e_i)=0$, $\Var(\e_i)=\sigma^2$.
\item $\Cov(\e_i,x_{ij})=0$ для всех $i$ и $j$
\item Вероятность полного ранга матрицы $X$ равна единице
\end{enumerate}

При выполнении этих предпосылок оценки МНК существуют с вероятностью 1 

Оценки состоятельны
\begin{proof}
Разложим $\hb$ в виде $\hb=(X'X)^{-1}X'y=(X'X)^{-1}X'(X\beta+\e)=\beta+(X'X)^{-1}X'\e$

Заметим, что $(X'X)^{-1}X'\e=\left(\frac{1}{n}X'X\right)^{-1}\frac{1}{n}X'\e$.

$\plim \left(\frac{1}{n}X'X\right)=Var(X_{i.})$

$\plim \frac{1}{n}X'\e=0$
\end{proof}




Сравнение двух парадигм

\begin{tabular}{c|cc}
 & детерминированные $X$  & случайные $X$ \\ 
\hline 
$\E(y_i)$ & разные, $X_{i.}\beta$ & одинаковые \\ 
$sVar(y)$ --- несмещенная оценка для $Var(y_i)$ & Нет & Да \\
 
\end{tabular} 

\section{Разное}
\begin{enumerate}

\item Гипотеза $H_0$ по-английски читается как <<H naught>>
\item При проверке гипотезы об адекватности регрессии НЕЛЬЗЯ писать $H_0: R^2=0$. 


Гипотезы имеет смысл проверять о ненаблюдаемых неизвестных константах. 
Проверить гипотезу о том, что $R^2=0$ легко. 
Для этого не нужно знать ничего из теории вероятностей, достаточно просто сравнить посчитанное значение $R^2$ с нулём. 

Более того, даже корректировка $\E(R^2)=0$ неверна. 
Случайная величина $R^2$ всегда неотрицательна, поэтому при любых разумных предпосылках на $\varepsilon$ окажется, что $\P(R^2>0)>0$. 
А это приведёт к тому, что $\E(R^2)>0$ даже если $Y$ никак не зависит от $X$.


Единственный правильный вариант --- $H_0: \beta_2=\beta_3=\ldots=\beta_k=0$ и $H_a: \exists i\geq 2 : \beta_i\neq 0$.

Можно добавить, что при построении регрессии $\hy=\hb_1$ величина $R^2$ тождественно равна нулю, вне зависимости от того, чему на самом деле равен $y$. Но эту гипотезу тоже не надо проверять, ведь мы это точно знаем.
\end{enumerate}



\section{Ridge/Lasso regression}

LASSO --- Least Absolute Shrinkage and Selection Operator. Метод построения регрессии, предложенный Robert Tibshirani в 1995 году.

Вспомним обычный МНК:
\begin{equation}
\min_{\beta} (y-X\beta)'(y-X\beta)
\end{equation}


LASSO вместо исходной задачи решает задачу условного экстремума:
\begin{equation}
\min_{\beta} (y-X\beta)'(y-X\beta)
\end{equation}
при ограничении $\sum_{j=1}^{k}|\beta_j|\leq c$.

\todo[inline]{Проверить! Нет ли у $\beta_1$ особого положения?}

Естественно, при больших значениях $c$ результат LASSO совпадает с МНК. Что происходит при малых $c$?


Для наглядности рассмотрим задачу с двумя коэффициентами $\beta$: $\beta_1$ и $\beta_2$. Линии уровня целевой функции --- эллипсы. Допустимое множество имеет форму ромба с центром в начала координат.


\todo[inline]{на картинке три $c$: очень большое --- дающиее мнк решение, меньше --- ненулевые $\beta$, маленькое --- одна из $\beta$  равна 0}


То есть при малых $c$ LASSO обратит ровно в ноль некоторые коэффициенты $\beta$.


Применим метод множителей Лагранжа для случая, когда ограничение $\sum_{j=1}^{k}|\beta_j|\leq c$ активно, то есть выполнено как равенство. 

\begin{equation}
L(\beta,\lambda)=(y-X\beta)'(y-X\beta)+\lambda \left(\sum_{j=1}^{k}|\beta_j| - c \right)
\end{equation}

Необходимым условием первого порядка является $\partial L/\partial \beta =0$. 
Это условие первого порядка не изменится, если мы зачеркнём $c$ в выражении. 
Таким образом мы получили альтернативную формулировку метода LASSO:
\begin{equation}
\min_{\beta} (y-X\beta)'(y-X\beta)+\lambda \sum_{j=1}^{k}|\beta_j|
\end{equation}

LASSO пытается минимизировать взвешенную сумму $RSS=(y-X\beta)'(y-X\beta)$ и <<размера>> коэффициентов $\sum_{j=1}^{k}|\beta_j|$.


Мы не будем вдаваться в численные алгоритмы, которые используются при решении этой задачи.


Ridge regression отличается от LASSO ограничением $\sum \beta_j^2\leq c$. 
Также как и LASSO Ridge regression допускает альтернативную формулировку:

\begin{equation}
\min_{\beta} (y-X\beta)'(y-X\beta)+\lambda \sum_{j=1}^{k} \beta_j^2
\end{equation}

Также как и LASSO Ridge regression тоже приближает значения коэффициентов $\beta_j$ к нулю. 
Принципиальное отличие LASSO и RR. 
В LASSO краевое решение с несколькими коэффициентами равными нулю является типичной ситуацией. 
В RR коэффициент $\beta_j$ может оказаться точно равным нулю только по чистой случайности. 


LASSO допускает байесовскую интерпретацию...

Предположим, что априорное распределение параметров следующее:

...


Тогда мода апостериорного распределения будут приходится в точности (?) на оценки LASSO.


\todo[inline]{Может ли появиться мультимодальность? В точности ли на моду или только примерно?}



\end{document}