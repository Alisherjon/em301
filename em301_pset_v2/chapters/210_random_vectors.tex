\section{Случайные вектора}

\begin{enumerate}
\item Пусть $y=(y_1, y_2, y_3, y_4, y_5)'$ --- случайный вектор доходностей пяти ценных бумаг. Известно, что $\E(y')=(5, 10, 20, 30, 40)$, $\Var(y_1)=0$, $\Var(y_2)=10$, $\Var(y_3)=20$, $\Var(y_4)=40$, $\Var(y_5)=40$ и
\[
\Corr(y)=\begin{pmatrix}
0 & 0 & 0 & 0 & 0 \\
0 & 1 & 0.3 & -0.2 & 0.1 \\
0 & 0.3 & 1 & 0.3 & -0.2 \\
0 & -0.2 & 0.3& 1 & 0.3 \\
0 & 0 & -0.2 & 0.3 & 1 
\end{pmatrix}
\]
С помощью компьютера найдите ответы на вопросы:
\begin{enumerate}
\item Какая ценная бумага является безрисковой?
\item Найдите ковариационную матрицу $\Var(y)$
\item Найдите ожидаемую доходность и дисперсию доходности портфеля, доли ценных бумаг в котором равны соответственно:
\begin{enumerate}
\item $\alpha=(0.2, 0.2, 0.2, 0.2, 0.2)'$
\item $\alpha=(0.0, 0.1, 0.2, 0.3, 0.4)'$
\item $\alpha=(0.0, 0.4, 0.3, 0.2, 0.1)'$
\end{enumerate}
\item Составьте из данных бумаг пять некоррелированных портфелей 
\end{enumerate}

\end{enumerate}