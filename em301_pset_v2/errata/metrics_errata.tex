\documentclass{article}
\usepackage[utf8]{inputenc}
\usepackage[russian]{babel}
\usepackage[parfill]{parskip}
\date{\today}
\title{Список опечаток к книжке \\
<<Эконометрика в задачах и упражнениях>>}

\newcommand{\reportedby}[2]{{\small [First reported by #1 on \mbox{#2}.]}}
\newcommand{\erratum}[1]{\subsubsection*{#1}}
% \erroronpage <page> <line info> <contributor> <date> 
\newcommand{\erroronpage}[4]{\textbf{Стр. #1, #2} (#3, #4)}



\begin{document}
\maketitle
%\begin{abstract} Список очепяток
%\end{abstract}

\section{Первое издание задачника}

Здесь номера страниц и задач относятся к первому печатному изданию задачника.




\erroronpage{14}{задача 2.6, пункт 10}{авторы}{15.10.2014}

Должно быть $\hat{\beta} = \frac{1}{2} \frac{y_n - y_1}{x_n - x_1} + \frac{1}{2n}  \left( \frac{y_1}{x_1} + \ldots + \frac{y_n}{x_n} \right) $

\erroronpage{41}{задача 3.24}{Анна Тихонова}{01.11.2014}

Вместо <<уведичилась>> должно быть <<увеличилась>>

\erroronpage{68-76}{задача 4.13}{Александр Левкун}{22.10.2014}

В условии должно быть $y_i=\beta_1+\beta_2 x_{i2} + \beta_3 x_{i3}+\varepsilon_i$

Вместо $\beta_1$ везде должно быть $\beta_2$.

Вместо $\beta_2$ везде должно быть $\beta_3$.

Вместо $x_1$ везде должно быть $x_2$

\erroronpage{119}{задача 8.2, условие}{авторы}{22.10.2014}

Должно быть: В модели $y_i=\beta_1 + \beta_2 x_i +\varepsilon_i$

\erroronpage{119}{задача 8.3, условие}{авторы}{22.10.2014}

Должно быть: В модели $y_i=\beta_1 + \beta_2 x_i +\varepsilon_i$

\erroronpage{122}{задача 8.8}{Анна Тихонова}{01.11.2014}

Пропущена запятая во фразе <<Область, в которой>>

\erroronpage{123}{задача 8.9}{Анна Тихонова}{01.11.2014}

Пропущена запятая во фразе <<Область, в которой>>

\erroronpage{124}{задача 8.10}{Анна Тихонова}{01.11.2014}

Пропущена запятая во фразе <<Область, в которой>>

\erroronpage{125}{задача 8.11}{Анна Тихонова}{01.11.2014}

Пропущена запятая во фразе <<Область, в которой>>

\erroronpage{126}{задача 8.12}{Анна Тихонова}{01.11.2014}

Пропущена запятая во фразе <<Область, в которой>>

\erroronpage{178}{устав проверки гипотез}{Анна Тихонова}{01.11.2014}

Пропущена запятая во фразе <<Область, в которой>>

\end{document}